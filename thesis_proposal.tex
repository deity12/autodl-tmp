% !TeX program = xelatex
\documentclass[12pt]{ctexart}
\usepackage{geometry}
\geometry{a4paper, margin=2.5cm}
\usepackage{graphicx}
\usepackage{hyperref}
\usepackage{amsmath}
\usepackage{booktabs}
\usepackage{enumitem}
\usepackage{caption}
\usepackage{float}
\title{基于量子-经典混合时序编码与动态LLM图谱融合的金融收益预测研究(QL-MATCC-GNN)}
\author{作者:\underline{待填} \\ 导师:\underline{待填}}
\date{\today}
\begin{document}
\maketitle
\begin{abstract}
本研究提出 QL-MATCC-GNN,一种将因果趋势-波动分解(MATCC)、高效因果 RWKV 时序编码、条件性变分量子电路(VQC)与基于 LLM 构建的动态图谱通过图神经网络(GNN)融合的混合架构,用于短期股票收益预测。方法在高波动子集上启用量子分支以增强表达能力,同时使用分层采样与 S\&P500 子集保证图谱质量。本文给出算法实现、实验设计及评估协议,并通过消融实验与分层评估验证各模块贡献。
\end{abstract}

\section{引言}
简要说明背景与研究动机:现代金融时序预测面临高噪声与结构性风险;同时 LLM 在文本关系抽取方面展示出较强能力,GNN 可用于跨标的信息聚合;量子机器学习(VQC)在高不确定性问题上可能带来表达能力优势。本课题旨在系统性地将上述技术结合并进行严格评估。

\section{相关工作}
略述 Transformer/RWKV、GNN(GAT)、LLM 信息抽取与 VQC 在机器学习中的应用与局限,突出本研究的跨范式集成与工程化实现。

\section{方法}
\subsection{总体架构}
QL-MATCC-GNN 为模块化流水线,包含数据处理、LLM 图谱构建、时序编码器(MATCC + RWKV)、量子-经典混合通道(条件 VQC)、图神经网络(GAT)与融合头。输入为每只股票过去 $T$ 天的特征序列,输出为对数收益率预测或截面排序分数。

\subsection{数据处理}
数据按 `Ticker` 做滑动窗口(序列长度 $T$),训练/测试按时间切分(80/20)。训练时拟合 `StandardScaler` 或 `RobustScaler`,并在标准化后计算波动率分位数作为量子门控阈值(默认 p70)。清洗步骤包含 Inf/NaN 处理、极端值裁剪与前向填充,详见代码 `paper/dataProcessed/dataset.py`。

\subsection{LLM 图谱构建}
使用高质量 prompt 批量抽取公司间关系,并进行别名规范化与分层采样,优先使用 S\&P500 子集以保证图谱质量。产生的邻接矩阵与 tickers 列表用于训练时对齐(`paper/dataProcessed/build_graph.py`)。

\subsection{时序模块:MATCC 与 RWKV}
MATCC 使用因果滑动平均分离趋势与波动,RWKV_TimeMixing 使用 JIT 加速的线性时间注意算子,采用完整的线性层以保证全量模型的表达能力(`paper/models/base_model.py`)。确保完全因果、无未来泄露的实现细节在代码中已处理。

\subsection{量子-经典混合通道}
对标准化波动率高于阈值的样本启用 VQC 分支:输入先经线性投影并用 sigmoid 映射到 $[0,\pi]$,再通过 PennyLane 实现的变分电路(8量子比特 + 4层 StronglyEntanglingLayers,提供256维希尔伯特空间),测量输出后投影回高维并乘以可学习缩放因子(初始值0.5)与残差连接以稳定训练;低波动样本使用纯经典 FFN 分支。实现见 `Quantum_ChannelMixing` 与 `VQC_Block`。

\subsection{图神经网络与融合}
基于2头稀疏 GAT 对诱导子图(batch 节点及 Top-K 邻居)做聚合,使用邻居索引缓存加速查找。双头设计在保持表达能力的同时提升每个头的维度。时序特征与图特征经门控融合(Gated Fusion)自适应合并,最终由融合头输出回归结果(详见 `paper/models/gnn_model.py`)。

\subsection{损失与训练策略}
主损失为 MSE,可选加入方向性软惩罚 $L_{sign}$:
\[\mathcal{L}=\frac{1}{N}\sum_i (y_i-\hat y_i)^2 + \alpha\cdot\frac{1}{N}\sum_i\max(0,-y_i\hat y_i).\]
训练采用 `AdamW`,支持差分学习率(量子参数使用较小 lr)、AMP、梯度裁剪与早停。训练脚本位于 `paper/training` 目录。

\section{实验设计}
\subsection{数据与基线}
主数据为 `data/processed/Final_Model_Data.csv`,主实验以 S\&P500 子集为对象。对照基线包括 LSTM/Transformer、RWKV(无量子)、RWKV+统计图谱+GNN。消融研究包括 w/o Quantum、w/o Graph、w/o MATCC、w/o MarketGuidance。

\subsection{评估协议}
指标包含 MSE、MAE、RMSE、R^2、方向准确率、IC(Pearson)与 RankIC(Spearman)。按日截面计算 IC 并求日均作为主要口径;并按波动率划分低/中/高组分别评估量子分支贡献。代码中评估实现见 `paper/evaluation/evaluate_all.py`。

\subsection{超参数与敏感性分析}
默认设置:$n_{emb}=256,n_{layers}=3,n_q=8,seq=30,batch=512,epochs=30,dropout=0.1,lr=3\times10^{-4}$。量子模块采用8量子比特与4层纠缠,提供256维希尔伯特空间以匹配经典FFN的表达能力。将研究 $q_{threshold}$(p60/p70/p80)、量子比特数 $n_q$、VQC 层数与量子缩放因子对性能的影响。

\section{时间计划}
详见开题草稿的 7 个月计划(数据准备、实现各模块、消融与写作)。

\section{资源与风险}
列出计算资源(48GB GPU 推荐)、PennyLane 依赖与可替代经典分支以应对量子模拟受限等风险与应对策略。

\section{参考文献}
\begin{thebibliography}{9}
\bibitem{vaswani2017} Ashish Vaswani et al., "Attention Is All You Need", NeurIPS 2017.
\bibitem{velickovic2018} Petar Veli\v{c}kovi\'c et al., "Graph Attention Networks", ICLR 2018.
\bibitem{pennylane} PennyLane Documentation, Xanadu.
\end{thebibliography}

\section*{附录:代码位置}
- 数据与预处理:`paper/dataProcessed/dataset.py`、`paper/dataProcessed/build_graph.py`。
- 模型实现:`paper/models/base_model.py`、`paper/models/gnn_model.py`。
- 训练与消融:`paper/training/train_full.py`、`paper/training/train_ablation.py`。
- 评估:`paper/evaluation/evaluate_all.py`。

\end{document}
